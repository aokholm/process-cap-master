\documentclass[aip,amsmath,reprint, author-year]{revtex4-1}
\usepackage{url}

\usepackage{hyperref}
\usepackage{graphicx} % for graphics
%\usepackage{listings} % for code listtings
%\usepackage{color}

%\setcitestyle{round, author-year}

\setcounter{page}{1}

%\bibliographystyle{aipauth4-1.bst}

\begin{document}

\begin{abstract}
Summery of literature on Process Capability Databases (PCDB). Why are PCDB's needed, what are the common configurations and what are the typical problems related to PCDB.
\end{abstract}

\title{Literature review of Process Capability Databases}
\author{Andreas Bruun Okholm, s082562\\
Mathias Rask Møller, s082536 }
\affiliation{Technical University of Denmark}
 
\date{\today}
\maketitle

%Introduction

\section{Introduction}
For all manufacturing processes there is variation in the dimensions of the produced parts.
Understanding process variation and creating designs that take into account variation decreases cost of manufactured goods and increase user satisfaction. Gathering, managing and delivering easy to understand process data posses a lot of challenges.  




\section{Process Capability Data}
There has been a significant amount of research, which shows use of process capability data (PCD) to set correct tolerances will reduce rework, cost, failure rate, assembly problems and improve product performance \citep{tata1999process}.

In the 90'ties several papers were published on methods to predict process variation using process models \citep{thornton2000use}. However creating process models can be very complicated and unpractical for many manufacturing process. 
Another option for acquiring process capability data is to measure the actual process performance and store it in a database. \cite{perzyk1998selection} describes such a database for storing general mechanical process capability data for the use of selecting optimal manufacturing methods.

Variance of a components dimension is usually the result of multiple processes. 
 \cite{kern2003forecasting} has developed a framework for a general process capability database that utilise a model of input and output variation of each (sub) manufacturing process. 

\subsection{Using Process Capability Data}
\cite{kern2003forecasting} identifies the following fields of study, where PCD is necessary for proper analysis: Design for Quality including Design for Six Sigma (DFSS), Quality Function Deployment (QFD), robust design, manufacturing variation modelling \& propagation, selective Assembly and tolerance allocation. 
Additionally Response Surface Methodology (RSM) and Statistical Process Control (SPC) does gather process data, which could be used in a PCD. 
The PCD is also necessary for the sub contractor to make qualified offers.


\subsection{Cost of manufacturing}
Including cost into the process capability database will make will make it possible to easily include the production price for the within design decisions \citep{perzyk1998selection, thornton2000use}. 
A general model for estimating the relationship between tolerance and price based on 5 different sources has been developed by \cite{sfantsikopoulos1990cost}.
It can be concluded that the relationship follows a simple general formula, but the parameter to relate tolerance to price for the specific manufacturing method needs to be estimated for every process. 




\section{Process capability databases in industry}
\cite{tata1999effective, tata1999process} found that major US companies within automotive, aerospace, military and consumer had created Process Capability Databases (PCDB) and were successfully using them for manufacturing control, but were unable to successfully utilise the PCDB for design. What hindered mechanical design from using PCD? 

\subsection{Key Challenges}
\textbf{Organisational.}  A PCDB is a cross departments part form usually involving manufacturing, quality control and mechanical design within the company. 
Creating a project which meets the interests of all department is challenging. 
Managers does not fully understand the need for a PCDB and the resources required to collect and analyse the information. 
Lack of company wide visions for the PCDB results in locally developed and maintained within a particular manufacturing location or departments. 
The use of PCD in design projects is not directly encouraged and rewarded. 

\textbf{User interface.} Creating efficient indexing systems that allows for unambiguous process, feature geometry, material and stock selection. The databases are typically not easily searchable which makes it very difficult to find the correct data and in many cases the data sought after data does not exists. 
The data is not presented in a way which is easily understandable by mechanical designers.

\textbf{Information Technology.} Creating a database thats globally accessible and accept inputs from multiple different clients and interface with existing systems is complex. 
Developing the indexing systems and user interface for multiple different uses cases is also resource demanding task.  

\subsection{Proposed Solutions}
In the book "Variation Risk Management" \cite{thornton2004variation} presents a variety variation management strategies to be used in a company. 
Key Characteristics (KC) can be a product, system, assembly, part or process characteristics for which the cost of variation is high and the variation is high. 
KC's should have most focus for optimal use of resources. 

It is individual for different companies what data should be stored in a PCDB, but a general list consist of material, stock, process, feature dimension, lower limit, upper limit, machine, operator number, batch size, mean, standard deviation for each record. Each record is a summation of multiple measurements made of a batch of parts \citep{thornton2004variation}.

Using graph theory a graphical display for manufacturing capability is developed, which provides quick overview \citep{thornton2000use}. 
One of the key views shows bias of a set of measurements along the horizontal axis and the standard deviation of the measurement set along the vertical axis. 
Confidence intervals in both bias and std. are shown for each measurement to visualise the statistical significance. 
A systems for automatically detecting and visualising trends in the data is also described.

\cite{kern2003forecasting} recognised that the success of the system relies on consistent and correct data input. 
A method to index design characteristics is developed to consistently assign correct key features on parts. Being one of the most elaborate works on PCDB the target seems to highly specialised productions companies. 





\section{Public Process Capability Data}
From our own search we haven't been able to find any public available process capability database. We have searched using Google, Google scholar and DTU library, using the search words: "process capability database", "Example process capability", "open process database", "Process variance database", "variance prediction database", "process variance table" without success. 

There exists some very general guide tables relating tolerance grade (IT-scale) and machining operation \citep{united1967preferred, american1978preferred}. It is however unclear how the data for the table has been generated.

In \cite[p. 715]{oberg2008machinery} there is given a figure showing the surface roughness from common production methods. From 50 Ra to 0.012. Again no source of the information is given. \\[1cm]

\section*{References}
\bibliography{../PCDBmasterBibliography/PCDB_Master_bib.bib}



\end{document}