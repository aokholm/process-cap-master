\documentclass[aip,amsmath,reprint, author-year]{revtex4-1}
\usepackage{url}

\usepackage{hyperref}
\usepackage{graphicx} % for graphics
%\usepackage{listings} % for code listtings
%\usepackage{color}

%\setcitestyle{round, author-year}

\setcounter{page}{1}

%\bibliographystyle{aipauth4-1.bst}

\begin{document}

\begin{abstract}
A presentation of a PCDB structure where organizational and management issues i taken in to account.
\end{abstract}

\title{PCDB Structure}
\author{Andreas Bruun Okholm, s082562\\
Mathias Rask Møller, s082536 }\email[E-mail me at: ]{aokholm@gmail.com}  \email[Email me at:] {or mathiasrask@gmail.com}
\affiliation{Technical University of Denmark}
 
\date{\today}
\maketitle

%Introduction

\section{Introduction}


\section{The Problem}

Reasons why PCDB are not more in use than it is today.

\begin{itemize}

\item{Design engineers are often unfamiliar with their firm’s capability to manufacture parts (Tata and Thornton 1999).}
\item{Making manufacturing data easy to retrieve for design engineers is difficult because it is often dispersed throughout an organization and can be in numerous unique forms, making it difficult to interpret. (Kern 2003).}
\item{lack of a company-wide vision for PCD usage and poor communication between manufacturing and design. (tata 1999)}
\item{Although companies have created process capability databases (PCDBs), the data is not being utilized by design. (tata 1999)}
\item{Design’s lack of trust and understanding of data. (tata 1999).}
\item{Lack of incentives for PCD use.(tata 1999).}



\end{itemize} 

\end{document}